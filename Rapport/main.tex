\documentclass[nn]{NMBU}

\usepackage{multirow}
\usepackage{float}
\usepackage{algorithm}
\usepackage[noend]{algpseudocode}

\credits{10}
\studyprogramme{Environmental Physics and Renewable Energy}
\course{INF202 Advanced Programming Project} % Include the name of the course
\courseyear{2024} % Include the year of the course

\setauthor{Jonas Kusch}
\settitle{Test title}

\begin{abstract}
Here I can write some text as I would do in word.
\end{abstract}

%\firstauthor{Firstname Othernames Lastname}

\begin{document}

\section{Introduction}

This here is my report for INF202. I am going to document the functionality of my software.

\subsection{Use of software development tools}
This is \textbf{now} the text that \textbf{I am writing} in \textit{the} software \underline{development section}. Now I want to tell you about the formula $f(x) = \int \alpha x_{2}^{9} \sin(2\pi) dx$.
\begin{align}\label{eq:g}
    g(x) = \frac{1}{2\pi} \sin(x)
\end{align}
Now I want to reference equation \eqref{eq:g}. You can also have an equation without a label. In this case we have
\begin{align*}
    h(x) =& \cos(x) \\
    f(x) = \sin(x)&
\end{align*}

\begin{align}
\vec{x} = 
    \begin{pmatrix}
        x_1 \\ x_2 \\ x_3
    \end{pmatrix}
\end{align}

\begin{enumerate}
    \item Some text for bullet point 1.
    \item Some text here as well.
\end{enumerate}

We have added a Figure \ref{fig:placeholder}.

\subsection{Linear algebra}

To define matrices and vectors, use the \texttt{pmatrix} environment. Use
\begin{align*}
    b = \begin{pmatrix} 
    1 \\ 2 \\ 3
  \end{pmatrix}
\end{align*}
for vectors. We can for example use
\begin{align}
    v(\vec{x}) = \begin{pmatrix}
        y-0.2 x \\
        -x
    \end{pmatrix}.
\end{align}

\section{Pseudo-Code}

You can include pseudo-code and reference it as Algorithm~\ref{alg:cap}.

\begin{algorithm}[H]
\caption{An algorithm with caption}\label{alg:cap}
\begin{algorithmic}
\Require $n \geq 0$
\Ensure $y = x^n$
\State $y \gets 1$
\State $X \gets x$
\State $N \gets n$
\While{$N \neq 0$}
\If{$N$ is even}
    \State $X \gets X \times X$
    \State $N \gets \frac{N}{2}$  \Comment{This is a comment}
\ElsIf{$N$ is odd}
    \State $y \gets y \times X$
    \State $N \gets N - 1$
\EndIf
\EndWhile
\end{algorithmic}
\end{algorithm}

The positioning of the pseudo-code and images can sometimes jump around.

\section{Images}
\label{ch:images}
To include images use
\begin{figure}
    \centering % This puts the image in the center
    \includegraphics[width=0.7\linewidth]{example_image.png}
    \caption{Caption}
    \label{fig:example_image}
\end{figure}

\begin{figure}
    \centering
    \includegraphics[width=0.5\linewidth]{Figures/ex2.png}
    \caption{This is the pixel image of a robot.}
    \label{fig:placeholder}
\end{figure}


You can also include multiple images:

\begin{figure}[H]
    \centering % This puts the image in the center
    \includegraphics[width=0.3\linewidth]{example_image.png}%
    \includegraphics[width=0.3\linewidth]{example_image.png}%
    \includegraphics[width=0.3\linewidth]{example_image.png}
    \caption{Three images with the same caption}
    \label{fig:example_image2}
\end{figure}

\section{Tables}
\label{ch:tables}

Tables should be used to for example define simulation parameters. Include tables in the following way:
\begin{center}
\begin{tabular}{ |c|c|c| } 
\hline
test & col2 & col3 \\
\hline
\multirow{3}{4em}{Multiple row} & cell2 & cell3 \\ 
& cell5 & cell6 \\ 
& cell8 & cell9 \\ 
\hline
Single row & cell10 & cell11 \\
\hline
a & b & c \\
\hline
\end{tabular}
\end{center}

% This is a comments

\begin{itemize}
\item \textbf{Context and Background}: Briefly introduce the topic, providing necessary background information.
\item \textbf{Problem Statement}: Clearly state the study problem or question the report addresses.
\item \textbf{Objectives}: Outline the aims or objectives of the study.
\item \textbf{Significance}: Explain the importance of the study and its potential impact.
\item \textbf{(Thesis or Hypothesis, \textit{if any})}: Present the hypothesis or the main argument of the report.
\end{itemize}

\section{Literature review}

For the literature review, you have to find reliable sources that you can use, such as scientific publications (`papers'). Try using \textit{Google Scholar} or simply \textit{Google} to locate those. \textbf{Talk with your potential supervisors and ask them for some references}.



\begin{itemize}
\item \textbf{Overview of Existing Research}: Summarise key studies related to your topic, highlighting major findings and trends.
\item \textbf{Theoretical and Conceptual Frameworks}: Discuss the theories and concepts that have been used in the literature to approach the problem.
\item \textbf{(Identification of Gaps, \textit{if present})}: Point out gaps, inconsistencies, or areas where further research is needed.
\end{itemize}

\section{Theory}

\begin{itemize}
\item \textbf{Theoretical Framework}: Outline the theories or models that guide your research.
\item \textbf{Conceptual Definitions}: Define key concepts and variables relevant to your study.
\item \textbf{Hypotheses}: If applicable, present specific hypotheses that stem from the theoretical framework.
\end{itemize}

\section{Methods}

\begin{itemize}
\item \textbf{Study Design}: Describe the overall research approach (e.g., qualitative, quantitative, experimental).
\item \textbf{Participants or Subjects}: Provide details on the sample size, selection criteria, and characteristics of participants.
\item \textbf{Materials and Instruments}: List tools, instruments, or materials used for data collection.
\item \textbf{Procedure}: Explain the steps of the research process, ensuring clarity on how the study was conducted.
\item \textbf{Data Analysis}: Detail the methods used to analyze the data, including any statistical or qualitative techniques.
\end{itemize}

\section{Results}

\textit{This section is not present in your report since you are expected to solely perform a literature review. Generally, your results would go here.}

\section{Discussion and conclusion}

\begin{itemize}
\item \textbf{Interpretation of Results}: Discuss the meaning of the results in relation to your hypothesis or research question.
\item \textbf{Implications}: Explore the broader implications of your findings for the field or practice.
\item \textbf{Limitations}: Acknowledge any limitations that may affect the validity or generalisability of the study.
\item \textbf{Future Research}: Suggest potential directions for further research.
\item \textbf{Final Summary}: Conclude by summarising the key findings and their relevance.
\end{itemize}

\section*{Todos for a perfect report}
A good report, a Master's Thesis or a PhD thesis, are documents that are published for posterity (your report will not be published, \textit{but you can use it for your thesis}). This means that it is important they are well structured, well organised, and devoid of typos and trailing comments.

\subsection{Figures and Tables}
When referring to tables and figures, do not write ``in the figure below'' or ``in the table above'', always refer to a figure or table using a \verb|\ref{}|. A few pointers:
\begin{itemize}
    \item Captions (\verb|\caption{...}|) for tables go \textit{above} the table; Captions for figures go \textit{below} the figures;
    \item The label of a figure/table 
    \item For each figure/table, the \verb|\label{...}| most be attached to the caption: \\ \verb|\caption{...}\label{...}|. \LaTeX~gets confused otherwise.
    \begin{itemize}
        \item Keep things organised: for figures/tables, use \verb|\label{fig:}|/\verb|\label{tab:}| and after the colon \verb|:| be explicit, so it is easy for you to remember which label refers to each figure/table.
    \end{itemize}
\end{itemize}

\subsection{References}
All the documents, scientific papers, reports, books, manuals, news articles, websites, government communications, and art exhibitions, need to be documented in the \textit{bibliography}. To cite any of these you can use \verb|\cite{}|. To help you organise your citations, use simple helpful citation keys, e.g.~  \verb|\cite{Jozefowicz2015empirical}| with the last name of the first author, the year, and the first (relevant) word of the title. 

We sometimes have to cite two scientific publications or two reports, or just two citations together in general. For that, you can use \verb|\cite{citation1, citation2}|, i.e., separate the citations with a comma, like~\cite{Beck2000application, Beck2005timeseries}.

When you are closely following a book to explain something, which is often the case in a theory section, you can write at the start of the section you are about to introduce: ``This section follows closely the reference book by Sumiyoshi Abe and Yuko Okamoto, \emph{Nonextensive Statistical Mechanics and Its Applications}~\cite{Abe2001nonextensive}.'' This is usually helpful in the theory section.

A Master's~\cite{LastName2045norwegian} or a PhD thesis~\cite{Temult2038binding} should include the name of the university wherein it was written, as well as the year. Moreover, it should include a URL to the work, when available.

Articles and proceedings are practically the same in science, but in your bibliography, you should use \verb|@article| for articles and \verb|@inproceedings| for proceedings. The work by Jozefowivz et al. (2015)~\cite{Jozefowicz2015empirical} is an example of a proceedings.

% Note that the appendices should come after the bibliography, not before.

\bibliography{bib} % this command prints out your bibliography



\end{document}
